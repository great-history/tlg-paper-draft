Next, we find the emergence of the helical edge states after pressurization, which is in phase A in Figure~\ref{fig:2}(c). 
To see it clearer, we show the profiles of both four-terminal longitudinal conductance and transverse resistance along the linecut at D = 0V/nm 
before and after pressurization in Figure~\ref{fig:3}(a) and ~\ref{fig:3}(b). 
It can be seen that at 1Gpa, there is a prominent nearly quantized plateau in the longitudinal conductance profile, which reaches to about $4\frac{e^2}{h}$ ( actual $4.1\frac{e^2}{h}$)
in a magnetic field ranging from 0.3T to 0.8T, the transverse resistance drops almost to zero under the same condition. 
This signature indicates two pairs of helical edge states in phase A.

However, this topological edge state can be destroyed by increasing B or D.
In addition, we extract the line profiles of Gxx at different B in the range of [0.40T, 0.60T], shown in Figure~\ref{fig:3}(c). 
It can be seen that when the displacement field is large enough, the nearly quantized Gxx goes through a drastic increase, 
indicating the state at D = 0 V/nm is protected by mirror symmetry. Therefore, it is the quantum parity hall state that is reported in \cite{stepanov2019quantum}. 
This state is a helical edge state, an analogy to QSHE. 
The origin of this state can be attributed to the crossing of the zeroth Landau level from the monolayer valence band $(m, 0)$ and the zeroth Landau level from the bilayer conduction band $(b, 0)$ at the sample boundary, 
forming four spin-degenerate counter-propagating helical edge modes while the bulk remains insulating, 
resulting in the quantization of the four-terminal longitudinal conductance $G_{xx}$ to $\frac{4e^2}{h}$, which can be verified by Landauer-Buttiker formalism. 
Since the monolayer-like Landau level and bilayer-like Landau level are only valid with the preservation of mirror symmetry, 
a relatively large displacement field can break this symmetry and hybridize these two types of Landau levels, which leads to the disappearance of the Quantum Parity Hall state.

In addition, the state in \cite{stepanov2019quantum} exists at a temperature of 0.4K. In contrast, the state in our sample emerges at a higher temperature, about 1.6K, 
showing a more robust behaviour against the temperature effect. On the other hand, the instability of this state before pressure indicates that some interlayer hopping parameters may play an important role in the robustness of this state. 
Indeed, according to \cite{stepanov2019quantum}, the ground state energy of QPH can be expressed as 
\begin{equation}
    E_{QPH}=-\frac{15}{4}\sqrt{\frac{\pi}{2}}\frac{e^2}{\epsilon l_B}-3\Delta_{mb}-\ldots \; \label{eq:qhe_ene}
\end{equation}
(\ldots omits other terms) 
where $Δ_{mb}$ is the energy scale, which characterizes the overlap between the valence band of the monolayer-branch and the conduction band of the bilayer-branch as shown in Figure~\ref{fig:3}(d). 
This expression tells that the larger the $Δ_{mb}$, the more stable the QPH state is. 
To have a more detailed analysis, we have to figure out how this parameter changes after pressurization in our device. 
By doing this, we must figure out how the interlayer hopping parameters are modified after pressurization.