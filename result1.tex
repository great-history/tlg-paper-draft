% Phase diagram of $\nu=0$ at small B
We first investigate the phase diagram of tri-layer ABA graphene around the charge neutrality point by measuring the four-terminal longitudinal conductance $G_{xx}(\frac{e^2}{h})$ 
under a small magnetic field and displacement field. Figure~\ref{fig:2}(a) and ~\ref{fig:2}(b) show this before and after pressure. 
As can be seen, the phase diagram at 1 GPa reveals a very different phase diagram than that at 0 GPa.

Firstly, the phase diagram at 0 GPa is similar to that observed in \cite{stepanov2019quantum}. 
The conductance around B = 0T and D = 0 V/nm is the highest because of the metallic property caused by the overlap between the valence and conduction bands. 
As |D| increases, opening a band gap at the charge neutrality point leads to a quick decrease of $G_{xx}$, indicating the high quality of our sample. 
In addition, along the D = 0 V/nm line profile, a local minimum of $G_{xx}$ emerges around B = 0.25T, indicated by the green arrows, surrounded by a high-conductance ring. 
This signature demonstrates that the nature of this state is different from that at higher B or zero B.
This phase diagram is similar to that of \cite{stepanov2019quantum}.
In the latter, this behaviour is interpreted as the emergence of “Quantum Parity Hall State” (QPH), which has nearly quantized four-terminal longitudinal conductance. 
However, here, we do not observe these well-defined features to claim its edge state configuration, so it should be regarded as an incipient of QPH, 
demonstrating it is not very stable at a temperature as high as T = 1.6K. In contrast, the temperature in \cite{stepanov2019quantum} is 0.4K.

However, at 1 GPa, the phase diagram becomes very different: the high-conductance state near B = 0T and D = 0V/nm has shrunk into a much smaller region, and more features appear as B and D changes. 
The first interesting feature is the appearance of spikes of longitudinal conductance $G_{xx}(B, D)$ within a small range of B $\in$ [0.3T, 0.8T], 
marked by the dashed-line trajectories of different colours in Figure~\ref{fig:2}(b). 
These features refer to the phase boundaries of first-order phase transitions, which are accompanied by a sudden change of charge distribution among several flavours \cite{zondiner2020cascade}, 
driven by the competition between displacement field and magnetic field. 
Similar transport signatures have been found in bilayer graphene and rhombohedral tri-layer graphene, 
but requires larger magnetic field \cite{geisenhof2021quantum, weitz2010broken, maher2013evidence, winterer2023ferroelectric}. 
We find that one of the phase boundaries displays a parabolic-like trajectory at low B (marked by a purple dashed line) and a linear trajectory at high B (marked by a green dashed line), 
another one also displays a linear behaviour (marked by an orange dashed line). 
These boundaries demonstrate the competition between different energy scales. 
Based on these phase boundaries and other features, we can divide the phase diagram at the charge neutrality point into several regions, 
and four of them are labelled as A, B, C and D, as shown in Figure~\ref{fig:2}(c).

The phase boundary between phase A and phase B can be explained by the competition between exchange interaction caused by B and layer polarization driven by D, 
since the exchange interaction proportional to $\frac{e^2}{l_B}\sim \sqrt{B}$ whereas the layer polarization characterized by $\Delta_1=\alpha dD$, 
with $\alpha\approx 0.3$ the empirical parameters describing the screening effect 
and $d\approx 0.3nm$ the nearest interlayer distance. Following the analysis of \cite{zhang2012hund, stepanov2019quantum}, 
one role of the magnetic field is to tune the exchange interaction strength, whereas the displacement field favours a layer polarization. 
It is reasonable that these energy scales drive the competition between phase A and phase B. 
In contrast, the phase boundary between phase C and phase B or phase D and phase B, displays a linear-shape trajectory, indicating a different mechanism from that between phase A and phase B. 
This linear dependence indicates that the energy scale associated with B should also be linear with B. 
From some theoretical work \cite{zhang2012hund}, the Zeeman energy $E_{ZM}$, the difference between intralayer and interlayer exchange interaction $\delta E_{ex}$ and electrostatic energy $E_H$ are all linear functions of B. 
Therefore, it is possible that these two-phase boundaries appear because of the complex competition between $\Delta_1$ and $E_{ZM}, \delta E_{ex}, E_H$.

To better manifest the distribution of electrons in spin and layer flavours, the possible state configurations of phase A/B/C/D are given in Figure~\ref{fig:2}(c).
It is well acknowledged that when D is large enough, the phase would be layer-polarized to minimize the ground state energy, mainly driven by the energy scale $\Delta_1=\alpha edD$. 
Therefore, phase C should be fully layer polarized and spin unpolarized. On the other hand, phase A and phase D should be both spin and layer unpolarized. 
As discussed below, phase A holds a non-trivial helical edge state, which is spin degenerate. 
Also, in the absence of displacement field D, the state should have mirror symmetry, which requires zero layer- polarization. 
An interesting problem is the configuration of phase B. The appearance of conductance spikes indicates a sudden change of distribution in spin and layer flavours either between phase A/D and phase B or between phase C and phase B. 
So, phase B may have some spin polarization. In addition, the boundary slope between phase B and phase D is almost twice that between phase B and phase C. 
This approximate 2:1 ratio may be related to the energy scale associated with the layer coupling. 
It is possible that the transition between phase A to phase B requires electron transfer between the two interlayer interfaces (middle layer to top layer and bottom layer to middle layer) 
whereas the transition between phase C and phase B only requires electron transfer between one interlayer interface (middle layer to top layer). 
Based on these, it is possible that phase B has layer anti-ferromagnetism and is partially layer-polarized, as shown in Figure~\ref{fig:2}(c).