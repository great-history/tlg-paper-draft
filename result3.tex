% 一定要引用 pablo / andrea 的几篇文章
In order to extract the value of hopping parameters, we perform the magneto-transport measurement at these two pressures in the absence of displacement field D, following similar procedures in \cite{taychatanapat2011quantum, campos2016landau}.
the Figure~\ref{fig:4}(a) and \ref{fig:4}(d) show the four-terminal longitudinal resistance $R_{xx}$ as a function of carrier density n and magnetic field B, named Landau Fan diagrams, 
which reflects the evolution of Landau levels in tri-layer ABA graphene as the magnetic field changes.
At a relatively large magnetic field (here 3T), a series of oscillations of broad minima, which are separated by broad maxima, 
start to appear as a pattern where each minimum represents the complete filling of one Landau level or several quasi-degenerate levels. 
The slope of each minimum track corresponds to a filling factor $\nu=n\frac{h}{eB}$, where e is the electron charge, h is the Planck's constant, and n is the Landau level index. 
Most of the visible minimum tracks were traced out, shown in Figure~\ref{fig:4}(b) and \ref{fig:4}(e).

In Figure~\ref{fig:4}(a) and \ref{fig:4}(d), there are several crossing points with the enhancement of Rxx which are caused by the crossing of zeroth Landau levels 
from the monolayer branch $(m, 0)$ and some lowest Landau levels from the bilayer branch $(b, i)$, 
and the positions of these crossing points can be used to determine the hopping parameters. 
The most visible crossing points are marked as cyan dot (P1), red dot (P2), and black dot (P3) in both Figures,
which can be identified by $(\nu, B)$, and they depend on the hopping parameters of tri-layer ABA graphene sensitively. 
It can be seen that the positions of P1, P2, and P3 in Figure~\ref{fig:4}(b) shift upward to higher fields than that in Figure~\ref{fig:4}(a), 
which indicates the modification of hopping parameters after pressure.

The six-band continuum model, which includes all the SWMC parameters $(\gamma_0, \gamma_1, \gamma_2, \gamma_3, \gamma_4, \gamma_5, \delta, \Delta_2)$ is used to reflect the modification of interlayer hopping strength qualitatively,
which can be determined by using the positions of crossing points as inputs. 
The SWMC parameters can be classified into two types: the intra-layer type $(\gamma_0)$, which are not influenced by inter-layer distance d and \
the inter-layer type $(\gamma_1, \gamma_2, \gamma_3, \gamma_4, \gamma_5, \delta, \Delta_2)$ which sensitively depends on d.
Here, the positions of P1, P2, and P3 were determined as (6, 4.60), (10, 2.40), (14, 1.80) in the Figure~\ref{fig:4}(a) and (6, 5.50), (10, 3.00), (14, 2.23) in the Figure~\ref{fig:4}(b). 
Then, we optimize these parameters to fit these positions with the consideration of disorder effect $\Gamma = 0.8$ meV. 
The fitting results are shown in Table~\ref{tab:table1}, and they were used to get the simulated Landau level spectra as shown in Figure~\ref{fig:4}(c) and ~\ref{fig:4}(f), 
the positions of Landau levels crossing points (marked by dots of the same colours as in Figure~\ref{fig:4}(a) and \ref{fig:4}(d)) match well with the experimental result.
The important parameter in formula~(\ref{eq:qhe_ene}), $\Delta_{mb}$ was found several meV larger after pressure: it is 14.3 meV at 0 GPa whereas 18.4 meV at 1GPa which has increased about 4 meV after pressure.
This expected result can explain the enhanced stability of the quantum parity hall state since its ground-state energy is greatly lowered.