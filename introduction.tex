%\tableofcontents
%========================Introduction====================
Regarding multilayer graphene systems, their rich physical properties make them widely regarded as a platform 
for studying unconventional properties of two-dimensional electron systems. 
Up to now, many intriguing physical phenomena has been found in multilayer graphene systems, ranging from single-particle to many-body physics, 
such as the Lifshitz transition of Fermi surface topology \cite{shi2018tunable, bao2011stacking}, 
Landau level physics in the single-particle picture \cite{taychatanapat2011quantum, campos2016landau, che2020substrate, shimazaki2016landau},
quantum hall ferromagnetism even fractionalization under a high perpendicular magnetic field 
\cite{zhang2012hund, young2012spin, liu2022visualizing, coissard2022imaging, hunt2013massive, yu2014hierarchy, hunt2017direct, lee2014chemical, lee2013broken, stepanov2016tunable, datta2017strong}, 
and various symmetry broken states near charge neutrality and under a relatively small magnetic field \cite{weitz2010broken, stepanov2019quantum, zibrov2018emergent, winterer2022spontaneous}. 
Besides these phenomena, insulating bulk states with non-trivial edge channels can emerge in various graphene systems under different conditions, including chiral and helical edge states 
\cite{zhang2011spontaneous, young2014tunable, veyrat2020helical, geisenhof2021quantum, maher2013evidence, spanton2018observation, stepanov2019quantum, winterer2023ferroelectric, han2023correlated}.

Among them, tri-layer ABA graphene has attracted widespread attention due to its unique band structure at low energy regime. 
As a highly tunable platform, it provides an excellent opportunity to study its single-particle band structure and many-body phenomenon. 
In previous studies, its electronic properties have been studied by tuning multiple control knobs like carrier density, displacement field, and magnetic field, 
which lead to many interesting discoveries in this system \cite{taychatanapat2011quantum, campos2016landau, shimazaki2016landau, che2020substrate, bao2011stacking, stepanov2016tunable, lee2013broken, stepanov2019quantum, zibrov2018emergent, datta2017strong}.
Notably, because the low-energy band structure of multilayer graphene is highly sensitive to the interlayer hopping strength, 
applying hydrostatic pressure to reduce the interlayer distance and further study its properties has become a highly promising research approach. 

In recent years, condensed matter physics has witnessed a growing interest in exploring the topologically non-trivial edge states in materials. 
As an extension of the chiral edge state, the helical edge state is a special boundary state featured by the non-dissipative channels with opposite motion directions. 
Helical conductors have been proposed to realize Majorana statistics for quantum computation \cite{qi2011topological, hasan2010colloquium, oreg2010helical, lutchyn2010majorana}.
Since its first theoretical proposed \cite{kane2005quantum, kane2005z}, multiple ways are developed to achieve helical edge states, 
including topologically non-trivial band structures protected by time-reversal symmetry, such as those found in QSHE systems \cite{bernevig2006quantum, roth2009nonlocal}. 
In addition, in those systems where the conduction band is energetically below the valence band, the introduction of a magnetic field can cause the crossing of “hole-like” Landau levels with “electron-like” Landau levels at the system's boundary, 
leading to the emergence of spin-degenerate helical edge states which are composed of opposite-propagating chiral edge states.
The advent of multilayer graphenes has provided a fertile platform for exploring different types of helical edge states 
because of the unique band structure near the charge neutrality point (CNP). 
Many experimental evidence has confirmed the existence of helical edge states in multilayer graphene systems, ranging from monolayer to tetralayer graphene
\cite{veyrat2020helical, young2014tunable, maher2013evidence, stepanov2019quantum, che2020helical}.
% \cite{veyrat2020helical, young2014tunable, maher2013evidence, li2019metallic, stepanov2019quantum, che2020helical}.

Here we report the dramatic change of phase diagram at the charge neutrality point and the emergence of a kind of helical edge state, named quantum parity hall effect, after applying 1.0 GPa hydrostatic pressure.
Tri-layer ABA graphene is one of the material platforms which can hold helical edge states owing to its unique band structure.
The low-energy band structure of tri-layer ABA graphene consists of a monolayer-like valence band and a bilayer-like conduction band in the absence of displacement field, which have an energy overlap, as shown in Figure~\ref{fig:1}(b). 
Both branches are protected by mirror symmetry and have odd(monolayer) or even(bilayer) parity. However, when a displacement field is applied, they mix due to the breaking of mirror symmetry.