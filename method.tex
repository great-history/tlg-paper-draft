%\tableofcontents
%========================Method====================

Here, we fabricate the tri-layer ABA device that has high mobility for observing its quantum hall effect, and we changed its band structure and phase diagram by exerting hydrostatic pressure out of the plane.
Our device is a high-quality dual-gate device with a Hall-bar geometry, with hexagonal boron nitride(h-BN) as the dielectric spacer and few-layer graphite as the top and bottom gates, 
which is stacked together using a dry-transfer method \cite{wang2013one}. 
In detail, the tri-layer ABA graphene is encapsulated by two hBN sheets whose geometric capacitances are $C_t, C_b$, respectively, in the unit of $V^{-1}·cm^{-2}$. 
Moreover, outside of hBN sheets, two graphite stacks act as top and bottom gates. 
By modulating the top gate voltage $V_t$ and bottom-gate voltage $V_b$, the carrier density $n(10^{12}cm^{-2})$ and out-of-plane displacement field $D(V/nm)$ can be tuned continuously, 
according to the relation:
\begin{equation}
    \begin{split}
        n &= C_t V_t + C_b V_b \\
        D &= \frac{1}{2\epsilon_0}(C_t V_t - C_b V_b)
    \end{split}
    \label{eq:nD_formula}
\end{equation}
with $\epsilon_0$ the vacuum permittivity. 
So, this configuration enables us to measure its transport properties with multiple in-situ tunable external parameters.

We also utilized hydrostatic pressure to reduce the interlayer distance and strengthen the interlayer coupling.
It can change the material structure of Van der Waals(VdW) material systems without enhanced disorder effect, thanks to their weak out-of-plane VdW bonding.
Previous work shows that hydrostatic pressure can compress the interlayer distance of graphene superlattice by $5\%$ with less than 2.5 GPa\cite{yankowitz2018dynamic} and can induce superconductivity in twisted bilayer graphene away from magic angle\cite{yankowitz2019tuning}. Therefore, hydrostatic pressure is a powerful tool for tuning the electronic structure effectively.
Here, we mainly investigate the change of the device under a 1.0 GPa hydrostatic pressure by comparing it to that in ambient pressure.

The high quality of this device can be verified by the dependence of longitudinal resistance as a function of n and D. 
An external out-of-plane displacement field opens a gap at the charge neutrality point, schematically shown in Fig.~\ref{fig:1}(b). 
Fig.~\ref{fig:1}(c) and (d) show the resistance of the device at zero magnetic fields before and after pressurization. 
It is found that, along the charge neutrality point, the resistance of both increases dramatically, which indicates the appearance of an electric-field-induced band gap. 
This signature reflects a good quality and low disorder of our device before and after pressurization, which enables us to study the sample's band structure using the Landau fan diagram 
under an out-of-plane magnetic field.

In this work, transport measurements were conducted on this tri-layer ABA graphene device before and after 1.0 GPa pressure at a temperature T = 1.6K, and our findings are three-fold. 
Firstly, we explore the phase diagram of the tri-layer ABA graphene at the charge neutrality point as a function of the magnetic field and displacement field, 
by making a slow scan of the displacement field around and magnetic field around. 
The one after pressure reveals a complex competition between different energy scales. 
Interestingly, we observe the signatures of some helical edge state after pressure, indicated by its nearly quantized four-terminal longitudinal conductance and near-zero transverse resistance. 
This state fades away when D is large enough, demonstrating that it is protected by mirror symmetry, which is broken under a finite D field. 
Secondly, in order to partially explain the observed phenomenon above, we perform the magneto-transport measurement to investigate how the pressure modifies the hopping parameters. 
Specifically, the Landau fan diagram shows that the crossing points among Landau levels undergo a significant shift, appearing at a larger magnetic field at 1.0 GPa pressure. 
This phenomenon indicates that the strength of interlayer hopping, alternatively, the low energy band structure, can be modified by pressure. 
Finally, our magneto-transport measurement finds some signatures of the four-fold quasi-degenerate zeroth monolayer-like Landau levels, which cannot be explained under the single-particle picture. 
These signatures include the complete splitting of these degenerate Landau levels at a relatively high magnetic field (about 10 T) before and after pressure. 
Besides, we found the Landau level splitting at the intermediate magnetic field only after pressurization. 
These two signatures indicate that the enlarged Landau level gaps, which are inconsistent with the single particle picture, should be associated with enhancing interlayer exchange interaction.
