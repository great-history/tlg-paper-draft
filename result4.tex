Within the single-particle picture, as shown in Figure~\ref{fig:4}(c) and \ref{fig:4}(f), the zeroth Landau levels of monolayer-like branch and some lowest Landau levels of bilayer-like branch are almost degenerate within the valley-spin subspace. 
With disorder effect ($\Gamma = 0.8$ meV), a set of four-fold Landau levels should be regarded as a four-fold degenerate Landau level, 
which shows in the Landau fan as a single minimum track instead of four splitting minimum tracks. 
However, the four-fold degeneracy of the zeroth Landau level of the monolayer-like branch can be lifted off completely at the relatively high field both in Figure~\ref{fig:4}(a) and ~\ref{fig:4}(d), 
and at even lower field in Figure~\ref{fig:4}(d). 

This splitting directly indicates that the Landau level gaps become larger to make each one isolated from others even with the disorder effect. 
This phenomenon can be explained by several possible scenarios: the enlarged single-particle gap at the large magnetic field, the suppression of the disorder effect, and enhanced electron-electron interaction. 
However, the first two scenes can be eliminated here with the analysis below. 
First,  the energy of the zeroth Landau levels of the monolayer-like branch is $- \frac{\gamma_2}{2} + \Delta_2$ at $K$ valley and $\delta - \frac{\gamma_5}{2} + \Delta_2$ at $K^{\prime}$ valley, which are not dependent on the magnetic field. 
This fact eliminates the first one since the Landau level gaps are unchanged. 
As for the second one, it is also unlikely since the bilayer-like Landau levels does not show a clear, completely splitting signature at the same magnetic field, 
even though the gap is larger, as shown in the simulation results. 
In addition, our simulation results show that the zeroth Landau levels of the monolayer-like branch are almost degenerate, so even does not take the disorder effect into account, 
the degeneracy should not be broken in the single-particle picture. 
Therefore, this scenario is not possible either. So only the last one, the effect of interaction, can be responsible for it. 
This phenomenon should be associated with the spontaneous symmetry-breaking caused by exchange interactions among Landau levels, 
named quantum hall ferromagnetism which beyond the single-particle picture\cite{zhang2012hund, lee2013broken, stepanov2016tunable, datta2017strong, young2012spin, liu2022visualizing, coissard2022imaging, lee2014chemical}. 

The characteristic energy scale of the exchange interaction between Landau levels can be estimated as $E_{ex} = \frac{e^2}{\epsilon l_B}\sim\sqrt B$, 
where $\epsilon\approx 6.6$ is the in-plan h-BN dielectric function.
At B = 10T, the strength of this interaction can be as large as 27 meV, which is much larger than the single-particle Landau level gaps within the spin-valley subspace. 
So, the electron-electron interaction should not be neglected at a relatively high field; it can mix all the possible single-particle states to form new states that spontaneously break the SU(4) symmetry. 
Moreover, the interaction scenario can also be used to explain the splitting happening at a lower field after pressure. 
The reduction of interlayer distance d can enhance the inter-layer exchange interaction, which may be important in tri-layer ABA graphene since the monolayer-like branch is layer-coherence.
Since the exchange interaction favours symmetry-breaking states, 
it is possible that the enhancement of inter-layer exchange interaction can help enlarge the gaps even in a relatively lower field.
However, there are some doubts about why this enhancement of inter-layer exchange interaction cannot enlarge the gaps in the bilayer-like branch. 
Maybe it also relates to the orbital nature of the zeroth Landau level in a monolayer-like branch, which is only associated with n=0,1 Landau levels.
